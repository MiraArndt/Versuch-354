\section{Theorie}
\label{sec:Theorie}

\subsection{Der ideale Schwingkreis}

In der Elektrotechnik setzt sich ein idealer Schwingkreis lediglich aus einem Kondensator mit der
Kapazität $C$ und einer Spule mit der Induktivität $L$ zusammen. Unter der Vorraussetzung, dass
diesem einmalig Energie hinzugeführt wurde, pendelt diese periodisch zwischen Kondensator und
Spule hin und her. Dies äußert sich darin, dass beispielsweise zu Beginn der Kondensator 
aufgeladen ist und sich zwischen den Kondensatorplatten ein elektrisches Feld bildet. Wandern 
die Elektronen nun von der einen Kondensatorplatte durch den Draht und damit auch mitunter durch 
die Spule, so nimmt mit dem elektrischen Feld auch die darin gespeicherte Energie ab. Allerdings
baut sich in der Spule ein magnetisches Feld auf. Da davon ausgegangen wird, dass ein verlustfreier
Schwingkreis vorliegt, muss Energieerhaltung gelten, was bedeutet, dass die Abnahme der im elektrischen
Feld gespeicherten Energie exakt der Zunahme der elektrischen Energie im magnetischen Feld entspricht.

\subsection{Der reale Schwingkreis}

Der Unterschied zwischen idealem und realem Schwingkreis liegt darin, dass bei einem realen 
Schwingkreis ein ohmscher Widerstand verbaut ist. Somit verliert der Schwingkreis mit jedem 
Durchgang Energie in Form von joulscher Wärme. In diesem Fall wird von einem gedämpften Schwingkreis
gesprochen.
Wie der Abbildung (Referenz) entnommen werden kann, bildet der RCL-Schwingkreis eine Masche. Nach
dem zweiten Kirchhoffschen Gesetz ist die Summe aller Spannungen in einer geschlossenen Masche 
gleich null.
Somit ergibt sich für den RCL-Kreis die Gleichung
\begin{equation}
    U_R(t)+U_C(t)+U_L(t) = 0.
    \label{eq:dgl}
\end{equation}
Dabei gelten für die Spannungen die Beziehungen
\begin{align}
    U_R(t) = R I(t) \\ 
    U_C(t) = \frac{Q(t)}{C} \\ 
    U_L(t) = L \frac{\mathrm{d}I}{\mathrm{d}t},
\end{align}
\noindent
wobei für die Stromstärke 
\begin{equation}
    I(t) = \frac{\mathrm{d}Q(t)}{\mathrm{d}t}
\end{equation}
\noindent
gelten soll.
Mithilfe dieser Zusammenhänge ergibt sich aus \ref{eq:dgl} die lineare Differentialgleichung zweiter Ordnung
\begin{equation}
    \ddot{I}(t) + \frac{R}{L}\dot{I}(t) + \frac{1}{RC}I(t) = 0,
\end{equation}
\noindent
welche sich mithilfe des Ansatzes
\begin{equation}
    I(t) = U e^(iwt)
\end{equation}
\noindent 
lösen lässt. Durch Einsetzen des Ansatzes ergibt sich schließlich die charakteristische Gleichung
\begin{equation}
    -w^2 + i\frac{R}{L}w+\frac{1}{LC} = 0.
\end{equation}
\noindent
Mit den beiden Kreisfrequenzen $w_1$ und $w_2$ 
\begin{equation}
    w_{1,2} = i \frac{R}{2L} \pm \sqrt{\frac{1}{LC}-\frac{R^{2}}{4L^{2}}}
\end{equation}
\noindent
kann die allgemeine Lösung der Differentialgleichung 
\begin{equation}
    I(t) = U_1 e^{i w_1 t} + U_2 e^{i w_2 t},
    \label{eq:lsg}
\end{equation}
\noindent
wobei $U_1$ und $U_2$ komplexe Koeffizienten darstellen. Wird nun die Darstellung
\begin{align}
    \mu = \frac{R}{4\pi L} \nonumber \\
\end{align}
\noindent
\begin{align}
    \nu = \frac{1}{2\pi} \sqrt{\frac{1}{LC}-\frac{R^{2}}{4L^{2}}} \nonumber \\
\end{align}
\noindent
verwendet, lässt sich \ref{eq:lsg} zu
\begin{equation}
    I(t) = e^{-2\pi \mu t} \left( U_1 e^{i 2 \pi \nu t} + U_2 e^{-i 2 \pi \nu t} \right)
\end{equation}
\noindent
umschreiben.





\cite{sample}
