\section{Diskussion}
\label{sec:Diskussion}
Das vom Oszilloskopen aufgenommene Bild \ref{fig:a1}
entspricht der erwarteten Form einer gedämpften Schwingung (siehe \ref{fig:gsk})
Allerdings fällt auf, dass die Abklingdauer viel zu groß ist.
Würde direkt der Betrag der Steigung als Abklingdauer angenommen werden,
so würde sich für $R_{eff}$ ein wahrscheinlicherer Wert von
\begin{equation*}
    R_{eff}=(678\pm19)\,\si{\ohm}
\end{equation*}
\noindent ergeben. Dieser Wert würde dem tatsächlichen Widerstandwert
von $R=(30,3\pm0,1)\,\si{\ohm} + 600\,\si{\ohm}$ eher entsprechen als der
in \ref{sec:dae} errechnete Wert.
-Problem mit Inversem erklären?
-Probelm, dass Innenwiderstand vielleicht nicht dem
unseres Gerätes entspricht


Bei der Messung der Phasenverschiebung wurden nicht
genug Werte aufgenommen um den nach Gleichung (REFERENZ)
erwarteten Verlauf eines Arcustangens zu erkennen.
Außerdem stechen einige Werte, wie der dritte Messwert
in Abbildung \ref{fig:c} ungewöhnlich hervor, was
wohl auf ein falsches Ablesen der Größen $a$ oder
$b$ zurückzuführen ist. Bei der Bestimmung
von $\omega_1$ und $\omega_2$ sind Ungenauigkeiten
dadurch entstanden, dass jeweils der Wert abgelesen wurde,
welcher am nähsten am gesuchten Vielfachen von $\pi$
liegt. Doch diese Ungenauigkeiten können nicht erklären,
warum vro allem $\omega_2=283\,\si{\kilo\hertz}$ so stark vom errechneten Wert
$\omega_2=(394\pm0,9)\,\si{\kilo\hertz}$ abweicht.
(VERWEIS AUF PROBLEM MIT INNENWIDERSTAND?)