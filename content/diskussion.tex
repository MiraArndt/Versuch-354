\section{Diskussion}
\label{sec:Diskussion}
Das vom Oszilloskopen aufgenommene Bild \ref{fig:a1}
entspricht der erwarteten Form einer gedämpften Schwingung (siehe \ref{fig:gsk})
Allerdings fällt auf, dass die Abklingdauer viel zu groß ist.
Würde direkt der Betrag der Steigung als Abklingdauer angenommen werden,
so würde sich für $R_{eff}$ ein wahrscheinlicherer Wert von
\begin{equation*}
    R_{eff}=(678\pm19)\,\si{\ohm}
\end{equation*}
\noindent ergeben. Dieser Wert würde dem tatsächlichen Widerstandwert
von $R=(30,3\pm0,1)\,\si{\ohm} + 600\,\si{\ohm}$ eher entsprechen als der
in \ref{sec:dae} errechnete Wert.


Außerdem kann der, bei den Rechnungen angenommene Innenwiderstand,
vielleicht nicht dem tatsächlichen, am verwendeten Gerät existierendem,
Innenwiderstand entsprechen. Der angenommene Innenwiderstand
von $R_I=600\,\si{\ohm}$ wurde \cite{sample353} entnommen.

Der Unterschied vom gemessenen Wert $R_ap$ zum gerechneten Wert
lässt sich zum einen durch den nicht in die Rechnung 
mit einbezogenen Innenwiderstand erklären, doch weitere Abweichungen
können auch durch  einen statistischen oder systematischen
Fehler entstanden sein.

Der Wert für die Resonanzüberhöhung $q$ weicht stark vom Theoriewert ab.
Es fällt jedoch auf, dass die beiden durch den Faktor $U_0=2,3\,\si{\volt}$
besser inneinander überführt werden können. Es ist also wahrscheinlich,
dass ein Fehler bei der Auswertung oder der entsprechenden Formel \ref{eq:b1}
vorliegt.


Bei der Messung der Phasenverschiebung wurden nicht
genug Werte aufgenommen um den nach Gleichung (REFERENZ)
erwarteten Verlauf eines Arcustangens zu erkennen.
Außerdem stechen einige Werte, wie der dritte Messwert
in Abbildung \ref{fig:c} ungewöhnlich hervor, was
wohl auf ein falsches Ablesen der Größen $a$ oder
$b$ zurückzuführen ist. Bei der Bestimmung
von $\omega_1$,$\omega_2$ und $\omega_{res}$ sind Ungenauigkeiten
dadurch entstanden, dass jeweils der Wert abgelesen wurde,
welcher am nähsten am gesuchten Vielfachen von $\pi$
liegt. Doch diese Ungenauigkeiten können nicht erklären,
warum vor allem $\omega_2=283\,\si{\kilo\hertz}$ und $\omega_{res}=239\,\si{\kilo\hertz}$
so stark von den errechneten Werten
$\omega_2=(394\pm0,9)\,\si{\kilo\hertz}$ und $\omega_{res}=(161,7\pm0,7)\,\si{\kilo\hertz}$
abweichen. Auch statistische Fehler können diese Abweichungen nicht erklären.
Am wahrscheinlichsten ist, dass entweder ein systematischer Fehler vorliegt,
oder dass die Auswertung bzw. Durchführung fehlerhaft ist. 
Es wurde nicht zwischen positiver und negativer Phasenverschiebung unterschieden,
wodurch sich die Werte für die Frequenzen wahrscheinlich nicht 
korrekt bestimmen ließen.
