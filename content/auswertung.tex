\section{Auswertung}
\label{sec:Auswertung}
\subsection{Dämpfung des elektischen Schwingkreises}
\label{sec:dae}
\begin{figure}[H]
  \centering
  \includegraphics[width=13cm]{content/Auswertung.jpg}
  \caption{Kondensatorspannung in Abhängigkeit der Zeit beim gedämpften Schwingkreis}
  \label{fig:a1}
\end{figure}

\begin{table}[H]
  \centering
  
  \csvreader[tabular=c|c,
  head=false, 
  table head= $t\:/\:\si{\micro\second}$ & $U_C\:/\:\si{\volt}$ \\
  \midrule,
  late after line= \\]
  {daempfung.csv}{1=\eins, 2=\zwei}{$\num{\eins}$ & $\num{\zwei}$}
  
  \caption{Kondensatorspannung in Abhängigkeit der Zeit beim gedämpften Schwingkreis}
  \label{tab:a}
\end{table}

\noindent Aus Abbildung \ref{fig:a1} wurden 12 Wertepaare
abgelesen und in Tabelle \ref{tab:a} aufgelistet. Um
die Abklingduaer $T_{ex}$ zu bestimmen wurde nun zunächst
eine lineare Ausgleichsrechnung durchgeführt. Dabei
wurde eine Lineare Ausgleichsgerade mit Matplotlib
erstellt, welche in Abbildung \ref{fig:b} zu sehen ist.
Die Steigung

\begin{equation*}
  m=(-0,01032\pm0,00029)\cdot 10^{-3}\,\si{\second}^{-1}
\end{equation*}

\noindent dieser Geraden entspricht dem Faktor im Exponenten in Gleichung \ref{eq:a1},
da $U_C(t)$ proportional zu $I(t)$ ist. Die Abklingdauer
kann nach Gleichung \ref{eq:a2} durch das 
Inverse des Betrags der Steigung als 

\begin{equation*}
  T_{ex}=(96,9\pm2,7)\cdot10^{3}\,\si{\second}
\end{equation*}

\noindent berechnet werden. Mit der Abklingdauer kann nun eine nichtlineare Ausgleichskurve
erstellt werden, die der Einhüllenden in Abbildung \ref{fig:a1}
entspricht. Die Ausgleichsgerade ist mit den abgelesenen
Messwerten in Abbilung \ref{fig:a} zu sehen.

\begin{figure}[H]
  \centering
  \includegraphics{build/plot1.pdf}
  \caption{Amplitudenmaximum der Kondensatorspannung eines RLC-Kreises in Abhängigkeit der Zeit}
  \label{fig:a}
\end{figure}

\begin{figure}[H]
  \centering
  \includegraphics{build/plot1-1.pdf}
  \caption{Linearisierte Darstellung der Amplitudenmaxima der Kondensatorspannung eines RLC-Kreises in Abhängigkeit der Zeit}
  \label{fig:b}
\end{figure}

\noindent Der Dämpfungswiderstand $R_{eff}$ ergibt sich duch
Zusammenhang \ref{eq:a2} als

\begin{equation*}
  R_{eff}=(0,07224\pm0,0021)\cdot10^{-3}\,\si{\ohm}
\end{equation*}

\noindent Die Abweichung zum eigentlichen Widerstandwert $R=(30,3 \pm 0,1)\,\si{\ohm}$
lässt sich nicht allein durch den Innenwiderstand des Generators erklären,
welcher $R_G=600\,\si{\ohm}$ beträgt, denn damit würde $R=(30,3\pm0,1)\,\si{\ohm} + 600\,\si{\ohm}$





\subsection{Aperiodischer Grenzfall}
Bei der Messung nach \ref{sec:ap}
ergab sich für $R_{ap}$ ein Wert von

\begin{equation*}
  R_{ap}=1,4\,\si{\kilo\ohm}
\end{equation*}

\noindent Mit Gleichung (REFERENZ) lässt sich $R_{ap}$ 
als

\begin{equation*}
  R_{ap}=(1,673\pm0,004)\,\si{\kilo\ohm}
\end{equation*}

\noindent berechnen. Beim Verglich mit dem Messwert fällt 
auf, dass sich die Abweichung durch den nicht beachteten
Innenwiderstand $R_G=600\,\si{\ohm}$ des Generators erklären lässt.




\subsection{Frequenzabhängigkeit der Kondensatorspannung}

\begin{table}[H]
  \centering
  
  \csvreader[tabular=c|c,
  head=false, 
  table head= $\omega\:/\:\si{\kilo\hertz}$ & $U_C\:/\:\si{\volt}$ \\
  \midrule,
  late after line= \\]
  {tabelle2.csv}{1=\eins, 2=\zwei}{$\num{\eins}$ & $\num{\zwei}$}
  
  \caption{Frequenzabhängigkeit der Kondensatorspannung bei einer erzwungenen Schwingung}
  \label{tab:b}
\end{table}

\begin{figure}[H]
  \centering
  \includegraphics{build/plot2.pdf}
  \caption{Frequenzabhängigkeit der Kondensatorspannung bei einer erzwungenen Schwingung}
  \label{fig:c}
\end{figure}


\noindent Die Messwerte aus Tabelle \ref{tab:b} werden
in Abbildung \ref{fig:c} dargestellt. Dabei ist $U_0$ 
der gemessene Wert $U_0=2,3\,\si{\volt} $ der Generatorspannung. Die Resonanzüberhöhung
lässt sich als

\begin{equation*}
  q=2,52
\end{equation*}

\noindent ablesen. Dieser lässt sich mit dem nach Gleichung
\ref{eq:b1} als 

\begin{equation*}
  q=0,9599\pm0,0024
\end{equation*}
\noindent errechneten Wert vergleichen. Für diesen wurde
nun $R=(271,6\pm0,1)\,\si{\ohm} + 600\,\si{\ohm}$ angenommen.
In Abbildung \ref{fig:g} ist der Bereich
um die Resonanzfrequenz linear dargestellt.
\begin{figure}[H]
  \centering
  \includegraphics{build/plot3.pdf}
  \caption{Frequenzabhängigkeit der Kondensatorspannung bei einer erzwungenen Schwingung in Nähe der Resonanz}
  \label{fig:g}
\end{figure}



\subsection{Frequenzabhängigkeit der Phasenverschiebung zwischen Kondensator- und Generatorspannung}

\begin{table}[H]
  \centering
  
  \csvreader[tabular=c|c|c|c,
  head=false, 
  table head= $\omega\:/\:\si{\kilo\hertz}$ & $a\:/\:\si{\micro\second}$ & $b\:/\:\si{\micro\second}$ & $\phi$ \\
  \midrule,
  late after line= \\]
  {tabelle1.csv}{1=\eins, 2=\zwei, 3=\drei, 4=\vier}{$\num{\eins}$ & $\num{\zwei}$ & $\num{\drei}$ & $\num{\vier}$}
  
  \caption{Frequenzabhängigkeit der Phasenverschiebung zwischen Kondensator- und Generatorspannung}
  \label{tab:g}
\end{table}

\begin{figure}[H]
  \centering
  \includegraphics{build/plot4.pdf}
  \caption{Frequenzabhängigkeit der Phasenverschiebung zwischen Kondensator- und Generatorspannung}
  \label{fig:j}
\end{figure}
\noindent Die Messwerte aus Tabelle \ref{tab:g} werden in Abbildung
\ref{fig:j} dargestellt. Da bei der Durchführung nicht zwischen
positiver und negativer Phasenverschiebung unterschieden wurde,
werden alle Werte positiv dargestellt, obwohl Gleichung (REFERENZ)
zeigt, dass die Funktion bei kleinen Frequenzen negative Werte annimmt.
In Abbildung \ref{fig:i} ist
der Bereich um die Resonanzfrequenz linear dargestellt.
Diese lässt sich an der Stelle $\pi/2$ als

\begin{equation*}
  \omega_{res}=239\,\si{\kilo\hertz}
\end{equation*}


\noindent ablesen. Im Vergleich zum, durch Gleichung \ref{eq:res},
gerechneten Wert

\begin{equation*}
  \omega_{res}=(161,7\pm0,7)\,\si{\kilo\hertz}
\end{equation*}

\noindent fällt auf, dass der Theoriewert weit unter dem gemessenen Wert liegt.
Auch hier wurde $R=(271,6\pm0,1)\,\si{\ohm} + 600\,\si{\ohm}$ angenommen.
\begin{figure}[H]
  \centering
  \includegraphics{build/plot5.pdf}
  \caption{Frequenzabhängigkeit der Phasenverschiebung zwischen Kondensator- und Generatorspannung in Nähe der Resonanz}
  \label{fig:i}
\end{figure}
\noindent $\omega_1$ liegt bei $\pi/4$ und entspricht
somit dem Messwert

\begin{equation*}
  \omega_1=195\,\si{\kilo\hertz}.
\end{equation*}


\noindent $\omega_1$ liegt bei $\pi \cdot 3/4$ und entspricht
somit dem Messwert

\begin{equation*}
\omega_2=283\,\si{\kilo\hertz}.
\end{equation*}
\noindent Diese Werte können mit den beiden aus Gleichung (REFERENZ)
errechneten Werten 

\begin{equation*}
  \omega_1=(145\pm0,4)\,\si{\kilo\hertz}
\end{equation*}
\begin{equation*}
  \omega_2=(394\pm0,9)\,\si{\kilo\hertz}
\end{equation*}

\noindent verglichen werden.


